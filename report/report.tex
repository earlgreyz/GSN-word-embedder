\documentclass{article}
\usepackage[utf8]{inputenc}
\usepackage{graphicx}
\graphicspath{ {./images/} }

\author{Mikołaj Walczak}
\title{Semantic relations in embeddings}

\begin{document}
  \maketitle
  \section{T-SNE visualization}
  \begin{figure}[ht]
    \includegraphics[width=\textwidth]{embeddings_500}
    \caption{T-SNE visualization of the first 500 words from the corpus}
    \label{fig:tsne-500}
  \end{figure}

  \section{Singular-Plural relation}
  By analizing the t-sne visualizations the most noticable relation at the first
  glance was how closely singular and plural forms of the same noun occurred on
  the graph. I have decided to analyze if we can extract a vector $\delta$ such that
  for any nouns \texttt{s} and \texttt{p} such that \texttt{p} is a plural
  form of a noun \texttt{s} the following equation holds:
  \begin{equation}
    s + \delta = p
  \end{equation}
  Where $s$ and $p$ denote embeddings for \texttt{s} and \texttt{p} accordingly.

  \subsection{Definitions}
  For convinience lets define a relation of being a plural
  \begin{equation}
    P = \{\langle s, p \rangle : \textrm{\texttt{p} is a plural of \texttt{s}} \}
  \end{equation}

  \subsection{Results}
  A set of words has been extracted to check if the relation holds. For each
  $\langle s, p \rangle \in P$ and $\langle s', p' \rangle \in P$ the cosine
  similarity has been calculated $\gamma(s' + p - s, p')$. Partial
  results can be seen in the table below (Table \ref{tab:sinpl_sim}).

  \begin{table}[ht]
  \center
    \resizebox{\textwidth}{!}{%
  \begin{tabular}{|c||c|c|c|c|c|c|}
  \hline
    & $\texttt{s}_i$/$\texttt{p}_i$ & \textit{strata/y} & \textit{miejsce/a} & \textit{umowa/y} & \textit{firma/y} & \textit{odprawa/y} \\ \hline \hline
    $\texttt{s'}_i$/$\texttt{p'}_i$ & $\delta$ & $\gamma(s_1 + \delta, p_1)$ & $\gamma(s_2 + \delta, p_2)$ & $\gamma(s_3 + \delta, p_3)$  & $\gamma(s_4 + \delta, p_4)$  & $\gamma(s_5 + \delta, p_5)$ \\ \hline
    \textit{strata/y} & $p'_1 - s'_1$ & 1 & 0.9832 & 0.9995 & 0.9969 & 0.9993 \\ \hline
    \textit{miejsce/a} & $p'_2 - s'_2$ & 0.9807 & 1 & 0.9933 & 0.9788 & 0.9983 \\ \hline
    \textit{umowa/y} & $p'_3 - s'_3$ & 0.9977 & 0.9762 & 1 & 0.9925 & 0.9991 \\ \hline
    \textit{firma/y} & $p'_4 - s'_4$ & 0.9978 & 0.9840 & 0.9988 & 1 & 0.9990 \\ \hline
    \textit{odprawa/y} & $p'_5 - s'_5$ & 0.9944 & 0.9890 & 0.9985 & 0.9878 & 1 \\ \hline
  \end{tabular}
  }
  \caption{Cosine similarity between plural and generated plural forms}
  \label{tab:sinpl_sim}
  \end{table}

  \subsection{Concerns}

  \begin{table}[ht]
  \center
    \begin{tabular}{|c|c|c|c|}
    \hline
    $i$ & $\texttt{s}_i$ & $\texttt{p}_i$ & $\gamma(s_i, p_i)$ \\ \hline
    0 & strata & straty & 0.9986 \\ \hline
    1 & miejsce & miejsca & 0.9864 \\ \hline
    2 & umowa & umowy & 0.9990 \\ \hline
    3 & firma & firmy & 0.9993 \\ \hline
    4 & odprawa & odprawy & 0.9994 \\ \hline
    \end{tabular}
  \caption{Cosine similarity between singular and plural forms}
  \label{tab:sinpl_sim_words}
  \end{table}

  For $\langle s, p \rangle \in P$ the words are syntactically similar and
  therefore close to each other, resulting in a difference vector $\delta = p - s$
  having short length. Good approximations of plural forms might be result of
  $\delta$ not affecting much the vector $s$, whcih is on its own very close to vector $p$
  (as shown in the Table \ref{tab:sinpl_sim_words}).

  Putting both tables together we can see that in most cases adding the difference
  vector puts the word further away from its plural form. Table \ref{tab:sinpl_diff}
  shows the difference $\gamma(s + \delta, p) - \gamma(s, p)$ for each $\delta = p' - s'$
  where $\langle s', p' \rangle \in P$.

  \begin{table}[ht]
  \center
    \resizebox{\textwidth}{!}{%
  \begin{tabular}{|c||c||c|c|c|c|c|}
  \hline
     & $\delta$ & $\gamma(s_1 + \delta, p_1)$ & $\gamma(s_2 + \delta, p_2)$ & $\gamma(s_3 + \delta, p_3)$  & $\gamma(s_4 + \delta, p_4)$  & $\gamma(s_5 + \delta, p_5)$ \\ \hline \hline
    $\gamma(s_1, p_1)$ & $p_1 - s_1$ & 0.0013 & -0.015 & 0.0009 & -0.001 & 0.0006 \\ \hline
    $\gamma(s_2, p_2)$ & $p_2 - s_2$ & -0.005 & 0.0135 & 0.0068 & -0.007 & 0.0118 \\ \hline
    $\gamma(s_3, p_3)$ & $p_3 - s_3$ & -0.001 & -0.022 & 0.0010 & -0.006 & 0.0009 \\ \hline
    $\gamma(s_4, p_4)$ & $p_4 - s_4$ & -0.001 & -0.015 & -0.004 & 0.0007 & -0.002 \\ \hline
    $\gamma(s_5, p_5)$ & $p_5 - s_5$ & -0.005 & -0.010 & -0.001 & -0.011 & 0.0006 \\ \hline
  \end{tabular}
  }
  \caption{Cosine similarity between plural and generated plural forms}
  \label{tab:sinpl_diff}
  \end{table}


  \section{Clusters}

  Different groups of words which have the same role in the sentence seem to
  form clustures. Calculating cosine similarity between all of the pairs confirm
  this prediction made based on the visualization.

  \subsection{Abbreviations}
  Common polish abbreviations such as institutions (MSZ), different company names (CAG, KLM)
  or even desease names (WZW) all fall into this category. Cosine similarity of
  a small sample can be seen in the Table \ref{tab:abbreviations}.
  \begin{table}[ht]
  \center
    \begin{tabular}{|c|c|c|c|c|c|c|}
    \hline
    & KLM & PiS & CAG & MSZ & ZOZy & WZW \\ \hline
    KLM & 1.0000 & 0.9960 & 0.9971 & 0.9966 & 0.9944 & 0.9970 \\ \hline
    PiS & 0.9960 & 1.0000 & 0.9954 & 0.9966 & 0.9966 & 0.9927 \\ \hline
    CAG & 0.9971 & 0.9954 & 1.0000 & 0.9922 & 0.9889 & 0.9958 \\ \hline
    MSZ & 0.9966 & 0.9966 & 0.9922 & 1.0000 & 0.9991 & 0.9886 \\ \hline
    ZOZy & 0.9944 & 0.9966 & 0.9889 & 0.9991 & 1.0000 & 0.9868 \\ \hline
    WZW & 0.9970 & 0.9927 & 0.9958 & 0.9886 & 0.9868 & 1.0000 \\ \hline
    \end{tabular}
  \caption{Cosine similarity of abbreviations}
  \label{tab:abbreviations}
  \end{table}

  \subsection{Foreign names}
  Vectors of names (Severus, George) and surnames (Snape, Fordon) borrowed
  from English language also seem to be close to each other. Cosine similarity of
  a small sample can be seen in the Table \ref{tab:foreign_names}.
  \begin{table}[ht!]
  \center
    \begin{tabular}{|c|c|c|c|c|c|c|}
    \hline
     & George & Severus & Orton & Snape & Edward & Fordon \\ \hline
    George & 1.0000 & 0.9967 & 0.9971 & 0.9950 & 0.9964 & 0.9969 \\ \hline
    Severus & 0.9967 & 1.0000 & 0.9972 & 0.9950 & 0.9940 & 0.9946 \\ \hline
    Orton & 0.9971 & 0.9972 & 1.0000 & 0.9982 & 0.9957 & 0.9965 \\ \hline
    Snape & 0.9950 & 0.9950 & 0.9982 & 1.0000 & 0.9962 & 0.9951 \\ \hline
    Edward & 0.9964 & 0.9940 & 0.9957 & 0.9962 & 1.0000 & 0.9989 \\ \hline
    Fordon & 0.9969 & 0.9946 & 0.9965 & 0.9951 & 0.9989 & 1.0000 \\ \hline
    \end{tabular}
  \caption{Cosine similarity of foreign names}
  \label{tab:foreign_names}
  \end{table}

  \section{Summary}
  For deeper relations of the word embeddings like in \textit{word2vec} the
  embedder model would have to be tweaked more precisely. However this simple
  implementation already created interesting results such as word clusters of
  semantically similar words.
\end{document}
